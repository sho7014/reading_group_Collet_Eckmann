\documentclass[a4paper,11pt,uplatex]{jsarticle}
\usepackage[chap=2]{../ex}

\begin{document}

\maketitle
\subsection*{正誤一覧}
\begin{itemize}
    \item p.~11, Sec.~2.3, para.~2: 

    誤: $h:\Omega \to \mathbb{R}^t$

    正: $h:\Omega \to \mathbb{R}$

    \item p.~15: 

    誤: $\varrho = \sqrt{u^2+v^2}$

    正: $\varrho = \sqrt{u^2+v^2}-1$

    \item p.~15: 

    誤: $\varrho + 0.4\cos(\varphi)$

    正: $0.25\varrho + 0.4\cos(\varphi)$

\end{itemize}

\subsection{2.5.}

($\Rightarrow$) ${\bf X}, {\bf Y}$ に対応する流れをそれぞれ ${\varphi}_t, {\psi}_t$ とする.流れの定義より,
\begin{equation}
    \dot{\varphi}_t({\bm x}) = {\bf X}({\varphi}_t({\bm x})), \quad \dot{\psi}_t({\bm x}) = {\bf Y}({\psi}_t({\bm x})) \label{eq: flow}
\end{equation}
を満たす.微分同相写像 ${\bf \Phi}$ についてこれらの流れが共役,つまり
${\bf \Phi} \circ {\varphi}_t ({\bm x}) = {\psi}_t \circ {\bf \Phi}({\bm x})$ と仮定する.この式の両辺を $t$ で微分すると,
\begin{align}
    \left[ {\mathrm D}{\bf \Phi}({\varphi}_t({\bm x})) \right]\dot{\varphi}_t({\bm x}) = \dot{\psi}_t \circ {\bf \Phi}({\bm x}) \nonumber \\
    \Leftrightarrow \left[ {\mathrm D}{\bf \Phi}({\varphi}_t({\bm x})) \right]{\bf X}({\varphi}_t({\bm x})) = {\bf Y}({\psi}_t \circ {\bf \Phi}({\bm x}))
\end{align}
$t=0$ を代入すると,${\varphi}_0, {\psi}_0$ は恒等写像なので,
\begin{equation}
    {\rm D}_x{\bf \Phi}\cdot {\bf X}({\bm x}) = {\bf Y}({\bf \Phi}({\bm x})) \label{diffeo25}
\end{equation}

($\Leftarrow$) 式 (\ref{diffeo25}) を満たす微分同相写像について,${\bm y}(t) = {\bf \Phi}\circ {\varphi}_t ({\bm x})$ とする.このとき
\begin{align}
    \dot{\bm y}(t) = D{\bf \Phi}({\varphi}_t({\bm x})){\bf X}({\varphi}_t({\bm x})) &= {\bf Y}\circ {\bf \Phi} \circ {\varphi}_t ({\bm x}) \quad (\because \ \text{eq.~}(\ref{diffeo25})) \nonumber \\
    &= {\bf Y}({\bm y}(t))
\end{align}
したがって ${\bm y}(t)$ は ${\bf \Phi}({\bm x})$ を初期条件とし,${\bf Y}$ をベクトル場とする微分方程式の時刻 $t$ における解である.つまり ${\bm y} (t)= {\psi}_t \circ {\bf \Phi}({\bm x})$ である.

\subsection{2.14.}
$(\varrho,\varphi,w)$ 座標系ではこのトーラスは
\begin{equation}
    (\varrho,\varphi,w)^\top = (0.8\sin{\vartheta},\varphi,0.8\cos{\vartheta})^\top
\end{equation}
となる.ソレノイド写像を ${\bm S}$ と表す.2 点 ${\bm p}_1 = (\varrho_1,\varphi_1,w_1)^\top, {\bm p}_2 = (\varrho_2,\varphi_2,w_2)^\top$ に対して,${\bm S}({\bm p}_1) = {\bm S}({\bm p}_2) $ とすると,第 2 成分より $\varphi_1=\varphi_2$.これを第 1, 3 成分より $\varrho_1=\varrho_2, w_1 = w_2$ を得る.したがって,ソレノイド写像は単射である.
\begin{equation}
    {\bm S}({\bm p})= (0.2\sin{\vartheta}+0.4\cos{\varphi},2\varphi,0.2\cos{\vartheta}+0.4\sin{\varphi})^\top
\end{equation}
であり,
\begin{align}
    S({\bm p})_1^2+S({\bm p})_3^2 = 0.2+0.16\sin{\vartheta}\cos{\varphi}+0.16\cos{\vartheta}\sin{\varphi} \le 0.52 < 0.64 = ({\bm p})_1^2+({\bm p})_3^2
\end{align}
したがってこの単射は strict.

\subsection{2.16.}
区分拡大写像 $f$ に対応する区間を $J_f := \{I_{f,j}\}_j = \{[a_{f,j},a_{f,j+1})\}_j$ とする.

\begin{lemma}\label{lem: expandingmap}\mbox{}\par 
2つの区分拡大写像 $f, g$ を考える.$g\circ f$ の相空間を $J_f$ の要素の部分区間になっているような有限個の区間の集まりに分割して,各区間上で $g\circ f$ が単調な $C^2$ 写像となっているようにできる. 
\end{lemma}
\begin{proof}\mbox{}\par
$f(I_{f,j})\subset \cup_{k\in K}I_{g,j'}$ をみたすような最小の $K$ を考える.
$I_{f,j}$ 上で $f(x) = a_{g,\min{K}+k'}$を満たす点を $b_{j,k'} \ (1 \le k' \le |K|-1)$ とする.$I_{f,j}$ での $f$ の単調性より,このような $b_{j,k'}$ は一意に定まる.$a_{f,j} = b_{j,0} < b_{j,1} < \cdots < b_{j,|K|} = a_{f,j+1}$ を満たすような有限列を考える.$L_{j,p} := [b_{j,p},b_{j,p+1})$ とすると,この区間において $f \circ g$ は単調な $C^2$ 写像の合成であるから,単調な $C^2$ 写像である.
\end{proof}
$f^n$ が区分拡大写像であると仮定する.補題 \ref{lem: expandingmap} より,$f^{n+1}$ の相空間を有限個の区間の集まりに分割して,各区間上で単調な $C^2$ 写像となるようにできる.また,
\begin{equation}
    |(f^{(n+1)m})'| \ge c^{n+1} >1
\end{equation}
がなりたつ.したがって $f^{n+1}$ は区分拡大写像である.帰納的に区分拡大写像の有限回の合成が区分拡大写像であることがいえる.


\end{document}