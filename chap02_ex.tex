\documentclass[a4paper,11pt,uplatex]{jsarticle}
\usepackage[chap=2]{ex}

\begin{document}

\maketitle
\subsection{2.5.}

($\Rightarrow$) ${\bf X}, {\bf Y}$ に対応する流れをそれぞれ ${\varphi}_t, {\psi}_t$ とする.流れの定義より,
\begin{equation}
    \dot{\varphi}_t({\bm x}) = {\bf X}({\varphi}_t({\bm x})), \quad \dot{\psi}_t({\bm x}) = {\bf Y}({\psi}_t({\bm x})) \label{eq: flow}
\end{equation}
を満たす.微分同相写像 ${\bf \Phi}$ についてこれらの流れが共役,つまり
${\bf \Phi} \circ {\varphi}_t ({\bm x}) = {\psi}_t \circ {\bf \Phi}({\bm x})$ と仮定する.この式の両辺を $t$ で微分すると,
\begin{align}
    \left[ {\mathrm D}{\bf \Phi}({\varphi}_t({\bm x})) \right]\dot{\varphi}_t({\bm x}) = \dot{\psi}_t \circ {\bf \Phi}({\bm x}) \nonumber \\
    \Leftrightarrow \left[ {\mathrm D}{\bf \Phi}({\varphi}_t({\bm x})) \right]{\bf X}({\varphi}_t({\bm x})) = {\bf Y}({\psi}_t \circ {\bf \Phi}({\bm x}))
\end{align}
$t=0$ を代入すると,${\varphi}_0, {\psi}_0$ は恒等写像なので,
\begin{equation}
    {\rm D}_x{\bf \Phi}\cdot {\bf X}({\bm x}) = {\bf Y}({\bf \Phi}({\bm x})) \label{diffeo25}
\end{equation}

($\Leftarrow$) 式 (\ref{diffeo25}) を満たす微分同相写像について,${\bm y}(t) = {\bf \Phi}\circ {\varphi}_t ({\bm x})$ とする.このとき
\begin{align}
    \dot{\bm y}(t) = D{\bf \Phi}({\varphi}_t({\bm x})){\bf X}({\varphi}_t({\bm x})) &= {\bf Y}\circ {\bf \Phi} \circ {\varphi}_t ({\bm x}) \quad (\because \ \text{eq.~}(\ref{diffeo25})) \nonumber \\
    &= {\bf Y}({\bm y}(t))
\end{align}
したがって ${\bm y}(t)$ は ${\bf \Phi}({\bm x})$ を初期条件とし,${\bf Y}$ をベクトル場とする微分方程式の時刻 $t$ における解である.つまり ${\bm y} (t)= {\psi}_t \circ {\bf \Phi}({\bm x})$ である.
\end{document}