\documentclass[a4paper,11pt,fleqn]{article}
\usepackage[chap=4]{../ex}

\begin{document}

\maketitle
\subsection{4.2.}
By taking derivatives of both sides of $f^{-1}\circ f(x) = x$, we obtain
\begin{equation}
    {\rm D}_{f(x)}(f^{-1})\cdot {\rm D}_x f = {\bf I}.
\end{equation}
\hruleskip

\subsection{4.5.}
Consider a characteristic polynomial 
\begin{equation}
    f(x) = x^2 - {\rm tr}(A)x + {\rm det}(A). 
\end{equation}
Its descriminant is ${\rm tr}(A)^2 - 4 {\rm det}(A) > 2^2-4 = 0$. 
Moreover, $f(0) = 1>0$ and $f(1) = 2-{\rm tr}(A) <0$. Hence the two eigenvalues $\lambda_1, \lambda_2$ of $A$ are real and positive and satisfy $\lambda_1 < 1 < \lambda_2$. 
\vskip 1.0em

The matrix $A$ can be diagonalized as $A = P{\rm diag}(\lambda_1,\lambda_2)P^{-1}$ since the eigenvalues are distinct. 
Let us define $y = P^{-1}x$. This conjugates $x\mapsto Ax$ and $y\mapsto {\rm diag}(\lambda_1,\lambda_2)y$. An orbit in the $y$-coordinate can be expressed as $(\lambda_1^ny_1(0), \lambda_2^ny_2(0))_{n\in \mathbb{Z}}, \ (y_1(0),y_2(0))\in \mathbb{R}^2$. The orbit belongs to $y_1y_2 = y_1(0)y_2(0)={\rm const}$, since $\lambda_1\lambda_2 = {\rm det}(A) = 1$. This is a hyperbola if $y_1(0)y_2(0) \neq 0$ and a line otherwise. An image of a linear transformation of a hyperbola (resp. a line) by a regular matrix $x = Py$ is a hyperbola (resp. a line). Thus each orbit of the linear map $x\mapsto Ax$ belongs to a hyperbola (or a line in a degenerate situation).  
\hruleskip

\subsection{4.10. (WIP)}
This answer greatly relies on~\cite{beyn1998numerical}. 
\vskip 1.0em

Let $f_1, f_2$ be $C^n$  
\if0
The Leibniz rule for the $n$-th derivative of their pointwise product $(f_1\cdot f_2)(x)$ is given as~\cite[Sec.~2.4]{abraham2012manifolds}\footnote{Note that $\left(\begin{array}{c}
    k \\
    i
\end{array}\right)$ in the last expression in the page.~96 is redundant.}
\begin{equation}
    D^n \left(f_1 \cdot f_2 \right)(x)\left(h_1, \ldots, h_n\right)=\sum_{\pi \in \mathcal{P}_n} \left(D^{\left|\pi_1\right|} f_1(x) h_{\pi_1} \cdot D^{\left|\pi_2\right|} f_2(x) h_{\pi_2}\right). \label{eq: leibniz}
\end{equation}
Here, for any finite subset $\omega = \{\omega_1, \cdots , \omega_n \}$ of $\mathbb{N}$, let $|\omega|$ denote its cardinality and 
\begin{equation}
    h_{\omega} = (h_{\sigma(\omega_1)},\cdots,h_{\sigma(\omega_n)}), \quad h_{\emptyset} = 1, 
\end{equation}
where $\sigma$ is a permutation and $\sigma(\omega_1)< \cdots < \sigma(\omega_n)$. Also, $\mathcal{P}_n$ is defined as 
\begin{align}
    \mathcal{P}_n=\left\{\pi=\left(\pi_1, \pi_2\right)\mid \pi_1, \pi_2 \subset\{1, \ldots, n\}, \pi_1 \cup \pi_2=\{1, \ldots, n\}, \pi_1 \cap \pi_2=\emptyset\right\}. 
\end{align}
Let 
\fi
and 
\begin{align}
    &\mathcal{Q}_k(\omega) = \left\{\pi=\left(\pi_1, \cdots, \pi_k \right)\mid 
    \emptyset \neq \pi_i \subset \omega, \ \bigcup_{i=1}^k \pi_i=\omega, \ \pi_i \cap \pi_j=\emptyset \
    \text{and} \ \min \pi_i<\min \pi_j \ \text{for all} \ i<j \right\}, \\
    &\mathcal{Q}(\omega) = \bigcup_{k=1}^{|\omega|}\mathcal{Q}_k(\omega), \quad \mathcal{Q}(\emptyset) = \{\emptyset\}, \quad \mathcal{Q}(n) = \mathcal{Q}(\{1,\cdots,n\}).  
\end{align}
Then the Fa\'{a} di Bruno's formula is given as~\cite[Sec.~2.4]{abraham2012manifolds}\footnote{Note that how the sum is taken in the last expression in the page.~97 is somewhat vague.}
\begin{equation}
    D^n\left(f_1 \circ f_2\right)(x) h_\omega =\sum_{\substack{\pi \in \mathcal{Q}(n) \\
    (k:=|\pi|)}} D^k f_1\left(f_2(x)\right)\left(D^{\left|\pi_1\right|} f_2(x) h_{\pi_1}, \ldots, D^{\left|\pi_k\right|} f_2(x) h_{\pi_k}\right),  \label{eq: bruno}
\end{equation}
where for any $\pi=\left(\pi_1, \cdots, \pi_k \right)$ let $|\pi|=k$ denote its length and for any finite subset $\omega = \{\omega_1, \cdots , \omega_n \}$ of $\mathbb{N}$, 
\begin{equation}
    h_{\omega} = (h_{\sigma(\omega_1)},\cdots,h_{\sigma(\omega_n)}) 
\end{equation}
where $\sigma$ is a permutation and $\sigma(\omega_1)< \cdots < \sigma(\omega_n)$.
\vskip 1.0em

Let $\tilde{g}(x):=(x,g(x))$. We take the $n$-th derivatives of the both sides of 
\begin{equation}
    g\circ f^{\rm s}(\eta,g(\eta)) = f^{\rm u}(\eta,g(\eta)) \Leftrightarrow g\circ f^{\rm s} \circ \tilde{g}(\eta) = f^{\rm u} \circ \tilde{g}(\eta)
\end{equation}
and apply the rule (\ref{eq: bruno}). 
\begin{align}
    &\text{(l.h.s):} \ D^n (g\circ f^{\rm s} \circ \tilde{g})(0)(h_1,\cdots,h_n) \nonumber \\
    & = \sum_{\substack{\pi \in \mathcal{Q}(n) \\
    (k:=|\pi|)}} D^k (g\circ f^{\rm s})\left(0\right)\left(D^{\left|\pi_1\right|} \tilde{g}(0) h_{\pi_1}, \ldots, D^{\left|\pi_k\right|} \tilde{g}(0) h_{\pi_k}\right) \nonumber \\
    & = \sum_{\substack{\pi \in \mathcal{Q}(n) \\
    (k:=|\pi|)}} \sum_{\substack{\tau \in \mathcal{Q}(k) \\
    (l:=|\tau|)}} D^l g (0) \left(D^{\left|\tau_1\right|} f^{\rm s}(0) \left(D^{\left|\pi_1\right|} \tilde{g}(0) h_{\pi_1}, \ldots, D^{\left|\pi_k\right|} \tilde{g}(0) h_{\pi_k}\right)_{\tau_1}
    , \right. \nonumber \\ 
    &\hspace{4cm} \ldots, \left. D^{\left|\tau_l\right|} f^{\rm s}(0) \left(D^{\left|\pi_1\right|} \tilde{g}(0) h_{\pi_1}, \ldots, D^{\left|\pi_k\right|} \tilde{g}(0) h_{\pi_k}\right)_{\tau_l}\right) \\
    &\text{(r.h.s):} \ D^n (f^{\rm u} \circ \tilde{g})(0)(h_1,\cdots,h_n) \nonumber \\
    &= \sum_{\substack{\pi \in \mathcal{Q}(n) \\
    (k:=|\pi|)}} D^k f^{\rm u}\left(0\right)\left(D^{\left|\pi_1\right|} \tilde{g}(0) h_{\pi_1}, \ldots, D^{\left|\pi_k\right|} \tilde{g}(0) h_{\pi_k}\right). 
\end{align}
Here, we defined  
\begin{equation}
    \left(D^{\left|\pi_1\right|} \tilde{g}(0) h_{\pi_1}, \ldots, D^{\left|\pi_k\right|} \tilde{g}(0) h_{\pi_k}\right)_{\omega} =
     \left( D^{\left|\pi_{\sigma(\omega_1)}\right|} \tilde{g}(0) h_{\pi_{\sigma(\omega_1)}},\cdots, D^{\left|\pi_{\sigma(\omega_n)}\right|} \tilde{g}(0) h_{\pi_{\sigma(\omega_n)}} \right),
\end{equation}
where $\sigma(\omega_1)< \cdots < \sigma(\omega_n)$. 
Thus we obtain 
\begin{equation}
    \begin{aligned}
        &\sum_{\substack{\pi \in \mathcal{Q}(n) \\
    (k:=|\pi|)}} \sum_{\substack{\tau \in \mathcal{Q}(k) \\
    (l:=|\tau|)}} D^l g (0) \left(D^{\left|\tau_1\right|} f^{\rm s}(0) \left(D^{\left|\pi_1\right|} \tilde{g}(0) h_{\pi_1}, \ldots, D^{\left|\pi_k\right|} \tilde{g}(0) h_{\pi_k}\right)_{\tau_1}
    , \ldots , \right. \\
    & \left. D^{\left|\tau_l\right|} f^{\rm s}(0) \left(D^{\left|\pi_1\right|} \tilde{g}(0) h_{\pi_1}, \ldots, D^{\left|\pi_k\right|} \tilde{g}(0) h_{\pi_k}\right)_{\tau_l}\right) = \sum_{\substack{\pi \in \mathcal{Q}(n) \\
    (k:=|\pi|)}} D^k f^{\rm u}\left(0\right)\left(D^{\left|\pi_1\right|} \tilde{g}(0) h_{\pi_1}, \ldots, D^{\left|\pi_k\right|} \tilde{g}(0) h_{\pi_k}\right).
    \end{aligned} \label{eq: manieq}
\end{equation}
\vskip 1.0em

We can draw a $n$-th order equation from (\ref{eq: manieq}) as follows:   

\bibliography{chap04}
\bibliographystyle{unsrt}
\end{document}