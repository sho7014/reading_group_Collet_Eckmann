\documentclass[a4paper,11pt,fleqn]{article}
\usepackage[chap=4]{../ex}

\begin{document}

\maketitle
\subsection{4.2.}
By taking derivatives of both sides of $f^{-1}\circ f(x) = x$, we obtain
\begin{equation}
    {\rm D}_{f(x)}(f^{-1})\cdot {\rm D}_x f = {\bf I}.
\end{equation}
\hruleskip

\subsection{4.5.}
Consider a characteristic polynomial 
\begin{equation}
    f(x) = x^2 - {\rm tr}(A)x + {\rm det}(A). 
\end{equation}
Its descriminant is ${\rm tr}(A)^2 - 4 {\rm det}(A) > 2^2-4 = 0$. 
Moreover, $f(0) = 1>0$ and $f(1) = 2-{\rm tr}(A) <0$. Hence the two eigenvalues $\lambda_1, \lambda_2$ of $A$ are real and positive and satisfy $\lambda_1 < 1 < \lambda_2$. 
\vskip 1.0em

The matrix $A$ can be diagonalized as $A = P{\rm diag}(\lambda_1,\lambda_2)P^{-1}$ since the eigenvalues are distinct. 
Let us define $y = P^{-1}x$. This conjugates $x\mapsto Ax$ and $y\mapsto {\rm diag}(\lambda_1,\lambda_2)y$. An orbit in the $y$-coordinate can be expressed as $(\lambda_1^ny_1(0), \lambda_2^ny_2(0))_{n\in \mathbb{Z}}, \ (y_1(0),y_2(0))\in \mathbb{R}^2$. The orbit belongs to $y_1y_2 = y_1(0)y_2(0)={\rm const}$, since $\lambda_1\lambda_2 = {\rm det}(A) = 1$. This is a hyperbola if $y_1(0)y_2(0) \neq 0$ and a line otherwise. An image of a linear transformation of a hyperbola (resp. a line) by a regular matrix $x = Py$ is a hyperbola (resp. a line). Thus each orbit of the linear map $x\mapsto Ax$ belongs to a hyperbola (or a line in a degenerate situation).  
\hruleskip

\subsection{4.10. (WIP)}

\end{document}