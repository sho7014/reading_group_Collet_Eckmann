\documentclass[a4paper,11pt,fleqn]{article}
\usepackage[chap=4]{../ex}

\begin{document}

\maketitle
\subsection{4.12.}
Inverse map is obtained as
\begin{equation}
    \left\lbrace\begin{array}{c}
        x' = x - 1.6y'(1-y'^2) \\
        y' = y + 1.6x(1-x^2)
    \end{array}\right. .
\end{equation}
The jacobian of $f$ is given as 
\begin{equation}
    Df(x,y) = \left(\begin{array}{cc}
        1 & 1.6(1-3y^2) \\
        -1.6(1-3x'^2) & 1-1.6^2(1-3y^2)(1-3x'^2)
    \end{array} \right). 
\end{equation}
$f$ is injective and $\mathop{\rm det}{(Df(x,y))}\equiv 1$, so $f$ is area-preserving. 
The fixed points $(x_*,y_*)$s satisfy simultaneous equations $y(1-y^2)=0, x(1-x^2)=0$, so $(x_*,y_*) \in \{0,1,-1\}^2$. If $\mathop{\rm tr}{(Df(x_*,y_*))}^2-4 \mathop{\rm det}{(Df(x_*,y_*))} = \mathop{\rm tr}{(Df(x_*,y_*))}^2 - 4 > 0$, $f$ is hyperbolic at the fixed point and non-hyperbolic otherwise. 
\begin{align}
    &\mathop{\rm tr}{(Df(x_*,y_*))}^2 - 4 = -1.6^2\times 4(1-3y_*^2)(1-3x_*^2) + 1.6^4(1-3y_*^2)^2(1-3x_*^2)^2 > 0 \nonumber \\
    &\Leftrightarrow g(x_*,y_*):= 4 -1.6^2(1-3y_*^2)(1-3x_*^2) \left\lbrace \begin{array}{cc}
        <0 & \text{if} \ |x_*|+|y_*|\neq 1 \\
        >0 & \text{otherwise}
    \end{array}\right. . \ (\because x_*, y_* \text{ are not irrational} )
\end{align}
We evaluate $g$ on the fixed points as 
\begin{align}
    g(0,0) = 4 - 1.6^2 >0 \nonumber \\
    g(0,\pm 1) = g(\pm 1,0) = 4+1.6^2\times2 >0 \nonumber \\
    g(\pm 1,\pm 1) = 4-4\times 1.6^2 <0 .
\end{align}
Therefore, $(0,0)$ is non-hyperbolic and the others are hyperbolic. 
\hruleskip

\subsection{4.20.}\label{ex: 420}
We take inner product with $\xi-\eta$ on both sides of $\xi=\eta+(\xi-\eta)$ as 
\begin{align}
    &(\xi-\eta,\xi) = (\xi-\eta,\eta)+||\xi-\eta||^2 \nonumber \\
    &\Leftrightarrow 2-2\cos{\theta} = ||\xi-\eta||^2. \label{eq: thetaineq}
\end{align}
Then 
\begin{align}
    &||A\xi - A\eta||^2 = ||A\xi||^2+||A\eta||^2-2(A\xi,A\eta) \ge ||A\xi||^2+||A\eta||^2-2||A\xi||\cdot||A\eta|| = (\varrho^{-1}-\varrho)^2 \\
    &||A\xi - A\eta||^2 \le ||A||_2^2||\xi-\eta||^2=||A||_2^2(2-2\cos{\theta}) \ (\because \text{Eq.~}(\ref{eq: thetaineq})). 
\end{align}
Thus the desired inequality is derived. 
\hruleskip

\subsection{4.21.}\label{ex: 421}
Let $v=\xi+\eta, \xi \in E^{\rm u}, \eta \in E^{\rm s}$. Let $\lambda_{\rm s}, C_{\rm s}$ (resp. $\lambda_{\rm u}, C_{\rm u}$) be constants which make all tangent vectors in $E^{\rm s}$ (resp. $E^{\rm u}$) stable and define $C = \max{(C_{\rm s},C_{\rm u})}$ and $\lambda = \max{(\lambda_{\rm s},\lambda_{\rm u})}$. Also let $P_{\rm s}, P_{\rm u}$ be projections onto $E^{\rm s}, E^{\rm u}$. 
\begin{align}
    &\langle v,v \rangle \ge ||\xi||^2 + ||\eta||^2 \ge ||\xi+\eta||^2 = (v,v), \\
    &\langle v,v \rangle \le \frac{C_{\rm s}^2||\xi||^2}{1-\lambda_{\rm s}^{2\epsilon}} + \frac{C_{\rm u}^2||\eta||^2}{1-\lambda_{\rm u}^{2\epsilon}} \le \frac{C^2}{1-\lambda^{2\epsilon}}(||\xi||^2 + ||\eta||^2) = \frac{C^2}{1-\lambda^{2\epsilon}}(||P_{\rm u}v||^2 + ||P_{\rm s}v||^2) \nonumber \\
    &\le \frac{C^2}{1-\lambda^{2\epsilon}}(||P_{\rm u}||_2^2 + ||P_{\rm s}||_2^2)(v,v). 
\end{align}
Thus we can take $D = \max{\left(1,\frac{C^2}{1-\lambda^{2\epsilon}}(||P_{\rm u}||_2^2 + ||P_{\rm s}||_2^2)\right)}$.
\hruleskip

\subsection{4.22.}
 Let $M=\max_{||x||\le \epsilon}{||D^2Q(x)||}, N = \max_{||x||\le \epsilon}{||D\varphi(x/\epsilon)||}$. These maxima exist because $D^2Q, D\varphi$ are continuous and the closed ball is compact. The Taylor theorem for $Q\in C^2$~[Sec.~4.5]\cite{zeidler2012applied} says that
\begin{align}
    &Q(x) = Q(0) + DQ(0)x + R_1(x) = R_1(x) \ (\because Q(0)=0, DQ(0)=0), \\
    &DQ(x) = DQ(0) + R_2(x) = R_2(x),
\end{align}
where $R_1, R_2$ are remainder terms. For $x$ in the closed ball of radius $\epsilon$, the mean value inequalities for the remainder terms~[Thm.~4.C]\cite{zeidler2012applied} lead to 
\begin{align}
    &||Q(x)|| = ||R_1(x)|| \le \frac{1}{2!}\sup_{0\le\tau\le 1}{||D^2Q(\tau x)(x,x)||}\le \frac{1}{2!}\sup_{0\le\tau\le 1}{||D^2Q(\tau x)}|| \cdot ||x|| \cdot ||x|| \le \frac{M}{2}||x|| \cdot ||x|| \nonumber \\
    & \le \frac{M\epsilon^2}{2} \\
    &||DQ(x)|| = ||R_2(x)|| \le \sup_{0\le\tau\le 1}{||D^2Q(\tau x)x||} \le M ||x|| \le M\epsilon.
\end{align}
The differential of $Q_\epsilon$ is 
\begin{equation}
    DQ_\epsilon(x) = \frac{1}{\epsilon}D\varphi(x/\epsilon)Q(x) + \varphi(x/\epsilon)DQ(x).
\end{equation}
Then 
\begin{equation}
    ||DQ_\epsilon(x)|| \le \frac{1}{\epsilon} || D\varphi(x/\epsilon) ||\cdot ||Q(x)|| + ||\varphi(x/\epsilon)||\cdot ||DQ(x)||
    \le \frac{M(N+2)}{2}\epsilon.
\end{equation}
\hruleskip

\subsection{4.23.}
($\mathscr{H}_\alpha$ is a normed vector space) If $\beta, \gamma \in \mathbb{R}$ and $f, g \in \mathscr{H}_\alpha$, Then
\begin{align}
    &||\beta f + \gamma g||_{\mathscr{H}_\alpha} = \sup_{x\neq y}\frac{||\beta(f(x)-f(y))+\gamma(g(x)-g(y))||}{||x-y||^\alpha} \le \sup_{x\neq y}\frac{\beta||f(x)-f(y)||}{||x-y||^\alpha} + \sup_{x\neq y}\frac{\gamma||g(x)-g(y)||}{||x-y||^\alpha} \nonumber \label{eq: holderbd} \\
    & < \infty. \\
    &\beta f(0) + \gamma g(0) = 0. 
\end{align}
So, $\beta f + \gamma g \in \mathscr{H}_\alpha$. Also, Eq. (\ref{eq: holderbd}) for $\beta=\gamma=1$ includes the triangle inequality. The other conditions to be a norm is trivial. Hence $\mathscr{H}_\alpha$ is a normed vector space. 
\vskip 1.0em

(Completeness) Consider a Cauchy sequence $\{f_n\}_{n=1}^\infty$ in $\mathscr{H}_\alpha$. This is also Cauchy in a space of continuous functions endowed with sup-norm, say, $BC(E)$. $BC(E)$ is complete, so the sequence has a limit in $BC(E)$. We denote the limit as $f$.

Every Cauchy sequence in a metric space is bounded, so $||f_n||_{\mathscr{H}_\alpha} \le M$ for all $n\in \mathbb{N}$ and some $M\in\mathbb{R}$. Hence
\begin{equation}
    ||f||_{\mathscr{H}_\alpha} = \lim_{n\to \infty}||f_n||_{\mathscr{H}_\alpha} \ (\because \text{the norm is continuous}) \le M < \infty. 
\end{equation}
Therefore $f \in \mathscr{H}_\alpha$. 

Next, we show $f_n\to f$ in $\mathscr{H}_\alpha$. For any $\epsilon >0$, there exists $N>0$ such that $||f_p-f_q||_{\mathscr{H}_\alpha}<\epsilon$ whenever $p, q \ge N$ since $\{f_n\}$ is Cauchy. Then for any $n\ge N$
\begin{align}
    ||f-f_n||_{\mathscr{H}_\alpha} = \lim_{m\to\infty}{||f_m-f_n||_{\mathscr{H}_\alpha}} \ (\because \text{ the norm is continuous}) \le \epsilon. 
\end{align}
Thus $f_n\to f$ in $\mathscr{H}_\alpha$. 

Therefore $\mathscr{H}_\alpha$ is complete. 
\hruleskip

\subsection{4.24.}
$R^{\rm u}(0,0)=0, R^{\rm s}(0,0)=0$ since $R \in \mathscr{H}_\alpha$. Hence
\begin{align}
    &\Psi^{\rm u}(R)(0,0)= (L^{\rm u})^{-1}R^{\rm u}(0,0) - (L^{\rm u})^{-1}Q_{\epsilon}^{\rm u}(0,0) = 0 \ (\because Q(0) = 0)  \nonumber \\
    &\Psi^{\rm s}(R)(0,0)= L^{\rm s}R^{\rm s}(0,0)+Q_{\epsilon}^{\rm s}(0,0) = 0.
\end{align}
\hruleskip

\subsection{4.25.}
The $C^2$ regularity is used to prove $||DQ_\epsilon||\le C\epsilon$. This inequality is used to bound $||\Psi(r)-\Psi(r')||_{\mathscr{H}_\alpha}$. To derive the inequality regarding $||\Psi(r)-\Psi(r')||_{\mathscr{H}_\alpha}$, $Q_\epsilon$ need not satisfy $||DQ_\epsilon||\le C\epsilon$. Instead, it is enough for $Q_\epsilon$ to be Lipschitz continuous of its Lipschitz constant $C\epsilon$.  
\vskip 1.0em

Let $\mathcal{B}_\epsilon$ be a closed ball of radius $\epsilon$ centered at the origin. Every continuous functions on a compact domain is globally Lipschitz, so $DQ$ is globally Lipschitz on $\mathcal{B}_\epsilon$. Let $K$ be the Lipschitz constant of $DQ$, then for all $x \in \mathcal{B}_\epsilon$ 
\begin{equation}
    ||DQ(x)|| = ||DQ(x) - DQ(0)|| \le K||x|| \le K\epsilon. 
\end{equation}
Let $L$ be the Lipschitz constant of $Q$ on $\mathcal{B}_\epsilon$, then~\cite[Lem.~3.1]{pathak2018introduction}
\begin{equation}
    L = \sup_{x\in\mathcal{B}_\epsilon}||DQ(x)||\le K\epsilon.
\end{equation}
For $x, y \in \mathcal{B}_\epsilon$, the mean value inequalities for the remainder terms lead to 
\begin{align}
    ||Q(x)||\le \sup_{0\le \tau \le 1}{||DQ(\tau x)x||}\le K\epsilon^2.
\end{align}
Let $M$ be the global Lipschitz constant of $\varphi$ on $\mathcal{B}_1$. $Q_{\epsilon}$ is zero outside the closed ball, so we can limit us to consider the ball to evaluate the Lipschitz constant. For $x,y \in \mathcal{B}_\epsilon$, 
\begin{align}
    &||Q_\epsilon(x)-Q_\epsilon(y)|| = ||\varphi(x/\epsilon)Q(x)-\varphi(y/\epsilon)Q(y)|| \nonumber \\
    &\le ||\varphi(x/\epsilon)-\varphi(y/\epsilon)||\cdot ||Q(x)|| + ||\varphi(y/\epsilon)||\cdot ||Q(x)-Q(y)|| \nonumber \\
    & \le \frac{M}{\epsilon}||x-y||K\epsilon^2+1\cdot K\epsilon||x-y||=K(M+1)\epsilon||x-y||
\end{align}
Thus $Q_\epsilon$ is Lipschitz continuous of its Lipschitz constant $C\epsilon$, where $C$ can be taken as $K(M+1)$. 
\hruleskip

\subsection{4.35.}
Let us assume $\Lambda$ is compact (at the beginning of this chapter, the compactness of $M$ is assumed but seems frequently violated (for example, $M=\mathbb{R}^n$)). In fact, in many literatures a uniformly hyperbolic set is defined to be compact. 
\vskip 1.0em

Let $||\cdot ||_{{\rm T}_xM}$ be an arbitrary norm and $||\cdot ||_{*, {\rm T}_xM}$ be is associated Lyapunov metric. Let $n_0\in \mathbb{N}$ such that $C\lambda^{n_0}<1$ in the Definition 4.~16 in the text\footnote{Note that $C>0$ is independent of $x \in \Lambda$ for a uniformly hyperbolic set. This seems implicitly assumed in the text. }. Then, as in the solution of \myref{ex: 420}, we obtain
\begin{equation}
    1-\cos{\theta_x} \ge \frac{(C^{-1}\lambda^{-n_0}-C\lambda^{n_0})^2}{||D_xf^{n_0}||_2^2}, \label{eq: angleineq}
\end{equation}
where $\theta_x$ is the angle between $E_x^{\rm s}$ and $E_x^{\rm u}$. $||D_xf^{n_0}||_2^2$ is bounded above on $\Lambda$ since $\Lambda$ is compact. Hence the r.h.s of~(\ref{eq: angleineq}) is uniformly bounded away from zero on $\Lambda$. Let $\mu \in (0,1)$ be a ($x$-independent) lower bound of the r.h.s.  
\vskip 1.0em

Let $v=\xi+\eta, \xi\in E_x^{\rm u}, \eta\in E_x^{\rm s}$. As in the solution of \myref{ex: 421}, 
\begin{equation}
    ||v||_{{\rm T}_xM} \le ||v||_{*, {\rm T}_xM}.
\end{equation}
Also, 
\begin{align}
    &||v||_{{\rm T}_xM}  = ||\xi||^2_{{\rm T}_xM}+||\eta||^2_{{\rm T}_xM}-2\cos{\theta_x}||\xi||_{{\rm T}_xM}||\eta||_{{\rm T}_xM} \nonumber \\
    &\ge ||\xi||^2_{{\rm T}_xM}+||\eta||^2_{{\rm T}_xM}-2(\mu -1)||\xi||_{{\rm T}_xM}||\eta||_{{\rm T}_xM} = (1-\mu)||\xi-\eta||^2_{{\rm T}_xM} + \mu(|\xi||^2_{{\rm T}_xM}+||\eta||^2_{{\rm T}_xM}) \nonumber \\
    &\ge \mu(|\xi||^2_{{\rm T}_xM}+||\eta||^2_{{\rm T}_xM}). \label{eq: ineqvvstdot}
\end{align}
Then 
\begin{align}
    &||v||_{*, {\rm T}_xM} \le \frac{C^2}{1-\lambda^{2\epsilon}}(||\xi||^2_{{\rm T}_xM}+||\eta||^2_{{\rm T}_xM})\le \frac{C^2}{\mu(1-\lambda^{2\epsilon})}||v||_{{\rm T}_xM}. \label{eq: ineqvvstdot2}
\end{align}
$C, \mu, \lambda, \epsilon$ are independent of $x\in \Lambda$. Hence, from (\ref{eq: ineqvvstdot}, \ref{eq: ineqvvstdot2}), the Lyapunov metric is uniformly equivalent to the original one. 
\hruleskip

\bib{chap04}
\end{document}