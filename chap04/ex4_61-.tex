\documentclass[a4paper,11pt,fleqn]{article}
\usepackage[chap=4]{../ex}

\begin{document}

\maketitle
\subsection{4.61.}
Let us define $\mathcal{B}, ||\cdot||_{\mathcal{B}}, A$ as in the proof of the Theorem~4.63 in the text. Let $U_{\epsilon'}$ be a $\epsilon'$-neighborhood of $\Lambda$. Then, for any ${\bf y} \in U_{\epsilon'}^{\mathbb{Z}}$, there exists ${\bf x} \in \Lambda^{\mathbb{Z}}$ such that
\begin{equation}
    ||{\bf y}-{\bf x}||_{\mathcal{B}} \le \epsilon'. \label{eq: closexy}
\end{equation} 
Suppose ${\bf y}$ is an $\epsilon'$-pseudo orbit lying in $U_{\epsilon'}$, i.e.,
\begin{equation}
    ||A{\bf y}-{\bf y}||_{\mathcal{B}} \le \epsilon'.
\end{equation}
$||Df||$ attains is maximum in $\Lambda$ since $Df$ is continuous and $\Lambda$ is compact. We denote the maximum as $M$. Then 
\begin{align}
    &\sup_{i\in\mathbb{Z}}{||f(x_i)-f(y_i)||} = \sup_{i\in\mathbb{Z}}{||f(x_i)-f(x_i-x_i+y_i)||} \nonumber \\
    &\le \sup_{i\in\mathbb{Z}}\sup_{\tau\in[0,1]}||Df(x_i)||\cdot||\tau(y_i-x_i)|| \ (\because \text{mean value inequality for the Taylor's theorem}) \le M\epsilon'. 
\end{align}
So 
\begin{align}
    ||A{\bf x}-{\bf x}||_{\mathcal{B}} = ||A{\bf x}-A{\bf y}+A{\bf y}-{\bf y}+{\bf y}-{\bf x}||_{\mathcal{B}} \le (M+2)\epsilon',  
\end{align}
namely, ${\bf x}$ is a $(M+2)\epsilon'$-pseudo orbit. 
Let 
\begin{equation}
    \epsilon = \min{\left(\frac{\delta }{2},\epsilon_{\frac{\delta}{2}}\right)},\quad \epsilon' = \frac{\epsilon}{M+2}, 
\end{equation}
where $\epsilon_{\frac{\delta}{2}}$ corresponds to the choice of $\epsilon$ for $\frac{\delta}{2}$ in the shadowing lemma. Then for any $\epsilon'$-pseudo orbit ${\bf y}$ lying in $U_{\epsilon'}$, there exists an $\epsilon$-pseudo orbit ${\bf x}$ lying in $\Lambda$, which satisfies (\ref{eq: closexy}) and can be $\delta/2$-shadowed by $z_0$. Obviously ${\bf y}$ is $\delta$-shadowed by $z_0$.  
\hruleskip

\subsection{4.64.}
From the chain rule, 
\begin{equation}
    {\rm D}_{x_{n-1}}{\rm D}_{x_{n-1-j}}f^j = {\rm D}_{x_{n-1-j}}f^{j+1}, \quad {\rm D}_{x_{n-1}}{\rm D}_{x_{n-1+j}}f^{-j} = {\rm D}_{x_{n-1+j}}f^{-j+1}. \label{eq: Dxchain}
\end{equation}
Hence
\begin{align}
    &\left[(I-D_{\bf x}A)\circ(L_{\bf x})v \right]_n = 
    \sum_{j=0}^{\infty} \mathrm{D}_{x_{n-j}} f^j \Pi_{E_{x_{n-j}}^{\mathrm{s}}} v_{n-j}-\sum_{j=1}^{\infty} \mathrm{D}_{x_{n+j}} f^{-j} \Pi_{E_{x_{n+j}}^{\mathrm{u}}} v_{n+j} \nonumber \\
    &- D_{x_{n-1}}f \left[ \sum_{j=0}^{\infty} \mathrm{D}_{x_{n-1-j}} f^j \Pi_{E_{x_{n-1-j}}^{\mathrm{s}}} v_{n-1-j}-\sum_{j=1}^{\infty} \mathrm{D}_{x_{n-1+j}} f^{-j} \Pi_{E_{x_{n-1+j}}^{\mathrm{u}}} v_{n-1+j} \right] \nonumber \\
    &=\sum_{j=0}^{\infty} \mathrm{D}_{x_{n-j}} f^j \Pi_{E_{x_{n-j}}^{\mathrm{s}}} v_{n-j}-\sum_{j=1}^{\infty} \mathrm{D}_{x_{n+j}} f^{-j} \Pi_{E_{x_{n+j}}^{\mathrm{u}}} v_{n+j} \nonumber \\
    &-\sum_{j=0}^{\infty} \mathrm{D}_{x_{n-1-j}} f^{j+1} \Pi_{E_{x_{n-1-j}}^{\mathrm{s}}} v_{n-1-j}+\sum_{j=1}^{\infty} \mathrm{D}_{x_{n-1+j}} f^{-j+1} \Pi_{E_{x_{n-1+j}}^{\mathrm{u}}} v_{n-1+j} \quad (\because \text{ Eq.~\ref{eq: Dxchain}}) \nonumber \\
    &=\Pi_{E_{x_{n}}^{\mathrm{s}}} v_{n} + \sum_{j=1}^{\infty} \mathrm{D}_{x_{n-j}} f^j \Pi_{E_{x_{n-j}}^{\mathrm{s}}} v_{n-j}-\sum_{j=1}^{\infty} \mathrm{D}_{x_{n+j}} f^{-j} \Pi_{E_{x_{n+j}}^{\mathrm{u}}} v_{n+j} \nonumber \\
    &-\sum_{j=1}^{\infty} \mathrm{D}_{x_{n-j}} f^{j} \Pi_{E_{x_{n-j}}^{\mathrm{s}}} v_{n-j}+\Pi_{E_{x_{n}}^{\mathrm{u}}} v_{n}+\sum_{j=1}^{\infty} \mathrm{D}_{x_{n+j}} f^{-j} \Pi_{E_{x_{n+j}}^{\mathrm{u}}} v_{n+j} = v_n. 
\end{align}
Thus $(I-D_{\bf x}A)\circ(L_{\bf x}) = {\rm id}$. 
\vskip 1.0em

$E^{\rm s}_x, E^{\rm u}_x$ are covariant w.r.t an action of the differential, i.e., ${\rm D}_{x_{n}}f^i(E^{\rm s}_{x_{n}})=E^{\rm s}_{x_{n+i}}, {\rm D}_{x_{n}}f^i(E^{\rm u}_{x_{n}})=E^{\rm u}_{x_{n+i}}$. Therefore 
\begin{equation}
    \Pi_{E_{x_{n-j}}^{\mathrm{s}}}{\rm D}_{x_{n-j-1}}f = {\rm D}_{x_{n-j-1}}f \ \Pi_{E_{x_{n-j-1}}^{\mathrm{s}}}, \quad \Pi_{E_{x_{n+j}}^{\mathrm{u}}}{\rm D}_{x_{n+j-1}}f = {\rm D}_{x_{n+j-1}}f \ \Pi_{E_{x_{n+j-1}}^{\mathrm{s}}}. \label{eq: prodcom}
\end{equation}
Also, 
\begin{equation}
    {\rm D}_{x_{n-j}}f^j{\rm D}_{x_{n-1-j}}f = {\rm D}_{x_{n-1-j}}f^{j+1}, \quad {\rm D}_{x_{n+j}}f^{-j}{\rm D}_{x_{n-1+j}}f = {\rm D}_{x_{n-1+j}}f^{-j+1}. \label{eq: DD}
\end{equation}
Then
\begin{align}
    &\left[(L_{\bf x})\circ (I-D_{\bf x}A)v \right]_n = 
    \sum_{j=0}^{\infty} \mathrm{D}_{x_{n-j}} f^j \Pi_{E_{x_{n-j}}^{\mathrm{s}}} v_{n-j}-\sum_{j=1}^{\infty} \mathrm{D}_{x_{n+j}} f^{-j} \Pi_{E_{x_{n+j}}^{\mathrm{u}}} v_{n+j} \nonumber \\
    &-  \sum_{j=0}^{\infty} \mathrm{D}_{x_{n-j}} f^j \Pi_{E_{x_{n-j}}^{\mathrm{s}}} {\rm D}_{x_{n-1-j}} v_{n-1-j} + \sum_{j=1}^{\infty} \mathrm{D}_{x_{n+j}} f^{-j} \Pi_{E_{x_{n+j}}^{\mathrm{u}}} {\rm D}_{x_{n-1+j}} v_{n-1+j}  \nonumber \\
    &=\sum_{j=0}^{\infty} \mathrm{D}_{x_{n-j}} f^j \Pi_{E_{x_{n-j}}^{\mathrm{s}}} v_{n-j}-\sum_{j=1}^{\infty} \mathrm{D}_{x_{n+j}} f^{-j} \Pi_{E_{x_{n+j}}^{\mathrm{u}}} v_{n+j} \nonumber \\
    &-\sum_{j=0}^{\infty} \mathrm{D}_{x_{n-1-j}} f^{j+1} \Pi_{E_{x_{n-1-j}}^{\mathrm{s}}} v_{n-1-j}+\sum_{j=1}^{\infty} \mathrm{D}_{x_{n-1+j}} f^{-j+1} \Pi_{E_{x_{n-1+j}}^{\mathrm{u}}} v_{n-1+j} \quad (\because \text{ Eqs.~\ref{eq: prodcom},~\ref{eq: DD}}) \nonumber \\
    &=v_n. 
\end{align}
Thus $(L_{\bf x})\circ(I-D_{\bf x}A) = {\rm id}$. 
\hruleskip

\subsection{4.70.}
Let $x_*, y_*$ are fixed points of $f, g$ respectively and $\Phi(x_*)=y_*$. Then
\begin{align}
    {\rm D}(\Phi\circ f)(x_*) = {\rm D}(g\circ \Phi)(x_*) \Leftrightarrow {\rm D}\Phi(x_*){\rm D}f(x_*)={\rm D}g(y_*){\rm D}\Phi(x_*) \ (\because f(x_*)=x_*, \Phi(x_*)=y_*).
\end{align}
Therefore ${\rm D}f(x_*)$ and ${\rm D}g(y_*)$ are similar. 
\vskip 1.0em

Let $x_*, y_*$ are fixed points of $f^n, g^n$ respectively and $\Phi(x_*)=y_*$. Then
\begin{align}
    {\rm D}(\Phi\circ f^n)(x_*) = {\rm D}(g^n\circ \Phi)(x_*) \Leftrightarrow {\rm D}\Phi(x_*){\rm D}f^n(x_*)={\rm D}g^n(y_*){\rm D}\Phi(x_*) \ (\because f^n(x_*)=x_*, \Phi(x_*)=y_*).
\end{align}
Therefore ${\rm D}f^n(x_*)$ and ${\rm D}g^n(y_*)$ are similar. 
\hruleskip

\subsection{4.76.}
We introduce a coordinate $z$ (resp. $w$) in direction of the long (resp. short) side of the red rectangle, where the unit length is defined as that of the long (resp. short) side. In the following, we consider in the $z-w$ coordinate. 
\vskip 1.0em

Let $R$ be the red rectangle and
\begin{equation}
    D_n = R\cap \left(\bigcap_{i=1}^n{f^{-i}(R)}\right), \quad D_0 = R.
\end{equation}
The set of points whose forward codes are the same is described as $\lim_{n\to\infty}{D_n}$. We can easily see that 
\begin{equation}
    D_{n+1} = R\cap f^{-1}(D_n). \label{eq: recdefD}
\end{equation}
Let $\lambda=(3-\sqrt{5})/2$, $T_{\rm L}=\lambda z\times[0,1)$, $T_{\rm R} =(\lambda z-(\lambda-1))\times[0,1)$. The recursive relation (\ref{eq: recdefD}) is translated as 
\begin{equation}
    D_{n+1}=T_{\rm L}(D_{n})\cup T_{\rm R}(D_{n}). 
\end{equation}
This decreasing series of set converges to the desired Cantor set. 
\hruleskip

\subsection{4.79.}
See~\cite[pp.~53--55]{bowen2008equilibrium}. \cite{rigg2021bowen} includes some follow-up explanations of~\cite{bowen2008equilibrium}, such as the proof of the proposition that $Z^*$ is open and dense. 
\hruleskip

\bib{chap04}
\end{document}